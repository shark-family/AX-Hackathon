\documentclass[a4paper, 11pt]{article}

% --- UNIVERSAL PREAMBLE BLOCK ---
\usepackage[a4paper, top=2.0cm, bottom=2.0cm, left=2.0cm, right=2.0cm]{geometry}
\usepackage{fontspec}

% 한글 및 다국어 지원
\usepackage[english]{babel}
\babelprovide[import, main, onchar=ids fonts]{korean}
\babelfont{rm}{Noto Sans}
\babelfont[korean]{rm}{Noto Sans CJK KR}

% 패키지 로드
\usepackage{enumitem}
\usepackage{xcolor}
\usepackage{graphicx}  % 이미지 삽입용
\usepackage{multirow}  % 표의 행(세로) 합치기용
\usepackage{array}     % 표 정렬 및 높이 조절용
\usepackage[hidelinks]{hyperref}
\usepackage[most]{tcolorbox}
\usepackage{tabularx}

% --- 색상 정의 ---
\definecolor{navyblue}{RGB}{0, 51, 102}
\definecolor{sectionbg}{RGB}{245, 247, 250}

% --- 커스텀 박스 스타일 정의 (Card Style) ---
\newtcolorbox{cvsection}[1]{
    breakable,
    enhanced,
    colback=white,
    colframe=navyblue,
    coltitle=white,
    fonttitle=\Large,
    title={#1},
    arc=2mm,
    boxrule=1pt,
    left=4mm, right=4mm, top=3mm, bottom=3mm,
    toptitle=2mm, bottomtitle=2mm,
    after skip=1.2em
}

% --- 문서 시작 ---
\begin{document}
\pagestyle{empty}

% [중요] 패키지 로드 부분에 \usepackage{tabularx} 와 \usepackage{array} 가 있어야 합니다.

% ====================================================================
% [헤더 영역]
% ====================================================================
\noindent
\begin{minipage}[c]{3.2cm}
    \centering
    \BLOCK{ if photo_path }
        % 사진 테두리 (선택사항)
        \setlength{\fboxsep}{0pt}
        \setlength{\fboxrule}{0.5pt}
        % [핵심] 너비 3cm, 높이 4cm로 강제 고정
        \fbox{\includegraphics[width=3cm, height=4cm]{\VAR{photo_path}}}
    \BLOCK{ else }
        % 사진 없을 때 빈 박스도 3cm x 4cm
        \framebox{\parbox[c][4cm][c]{3cm}{
            \centering\small\color{gray}사진\\(3cm$\times$4cm)
        }}
    \BLOCK{ endif }
\end{minipage}%
\hfill
% [오른쪽] 정보 테이블
\begin{minipage}[c]{\dimexpr\textwidth-3.5cm\relax}
    % [높이 맞춤 핵심]
    % 사진 높이(4cm)에 맞추기 위해 행 높이를 2.2배로 설정 (4행 기준 약 4cm)
    \renewcommand{\arraystretch}{2.2}
    
    % 세로 중앙 정렬(m) 및 가로 중앙 정렬(Y) 설정
    \renewcommand{\tabularxcolumn}[1]{m{#1}}
    \newcolumntype{Y}{>{\centering\arraybackslash}X}
    
    \begin{tabularx}{\linewidth}{|c|Y|c|Y|}
        \hline
        \textbf{성명} & \VAR{name} & \textbf{생년월일} & \VAR{birthdate} \\ \hline
        % 주소: 긴 내용 줄바꿈 허용 (너비 9.5cm)
        \textbf{주소} & \multicolumn{3}{>{\centering\arraybackslash}m{9.5cm}|}{\VAR{address}} \\ \hline
        \textbf{연락처} & \VAR{phone} & \textbf{이메일} & \small \VAR{email} \\ \hline
        \textbf{비상연락} & \multicolumn{3}{>{\centering\arraybackslash}m{9.5cm}|}{\VAR{emergency_contact}} \\ \hline
    \end{tabularx}
\end{minipage}
% ====================================================================
% [섹션 2: 경력 사항]
% ====================================================================
\begin{cvsection}{경력 사항}
    \BLOCK{ for job in experience }
    \noindent 
    \begin{minipage}[t]{0.75\textwidth}
        \textbf{\large \VAR{job.company}} \\
        \textit{\small \VAR{job.role}}
    \end{minipage}%
    \hfill
    \begin{minipage}[t]{0.20\textwidth}
        \raggedleft
        \textbf{\textit{\small \VAR{job.period}}} \\
        \textbf{\textit{\small \VAR{job.location}}}
    \end{minipage}

    \BLOCK{ if not loop.last }
    \vspace{0.8em}
    \hrule
    \vspace{0.8em}
    \BLOCK{ endif }
    \BLOCK{ endfor }
\end{cvsection}

% ====================================================================
% [섹션 3: 학력]
% ====================================================================
\begin{cvsection}{학력}
    \BLOCK{ for edu in education }
    \noindent
    \begin{minipage}[t]{0.75\textwidth}
        \textbf{\large \VAR{edu.institution}} \\
        \BLOCK{ if edu.degree }
            \textit{\small \VAR{edu.degree}}
        \BLOCK{ endif }
    \end{minipage}%
    \hfill
    \begin{minipage}[t]{0.20\textwidth}
        \raggedleft
        \textbf{\textit{\small \VAR{edu.period}}}
    \end{minipage}

    \BLOCK{ if not loop.last }
        \vspace{0.8em}
        \hrule
        \vspace{0.8em}
    \BLOCK{ endif }
    \BLOCK{ endfor }
\end{cvsection}


% ====================================================================
% [섹션 4: 자격증 및 어학
% ====================================================================
\begin{cvsection}{자격증 및 어학}
    % [표 형태로 변경]
    % 구조: 자격증명(6cm) | 세부내용(자동채움) | 날짜(우측정렬)
    \renewcommand{\arraystretch}{2} % 행 간격 조절
    \begin{tabularx}{\linewidth}{@{} p{6cm} X r @{}}
    \BLOCK{ for cert in certifications }
        \textbf{{\VAR{cert.title}}} & 
        \BLOCK{ if cert.score }
            \textbf{{\small \VAR{cert.score}}}
        \BLOCK{ elif cert.issuer }
            \textbf{{\small \VAR{cert.issuer}}}
        \BLOCK{ endif }
        & 
        {\small\color{black}\textbf{\textit{\VAR{cert.date}}}} \\
    \BLOCK{ endfor }
    \end{tabularx}
\end{cvsection}

\end{document}