\documentclass[a4paper, 11pt]{article}

% --- UNIVERSAL PREAMBLE BLOCK ---
\usepackage[a4paper, top=2.0cm, bottom=2.0cm, left=2.0cm, right=2.0cm]{geometry}
\usepackage{fontspec}

% 한글 및 다국어 지원
\usepackage[english]{babel}
\babelprovide[import, main, onchar=ids fonts]{korean}
\babelfont{rm}{Noto Sans}
\babelfont[korean]{rm}{Noto Sans CJK KR}

% 패키지 로드
\usepackage{enumitem}
\usepackage{xcolor}
\usepackage{graphicx}  % 이미지 삽입용
\usepackage{multirow}  % 표의 행(세로) 합치기용
\usepackage{array}     % 표 정렬 및 높이 조절용
\usepackage[hidelinks]{hyperref}
\usepackage[most]{tcolorbox}
\usepackage{tabularx}

% --- 색상 정의 ---
\definecolor{navyblue}{RGB}{0, 51, 102}
\definecolor{sectionbg}{RGB}{245, 247, 250}

% --- 커스텀 박스 스타일 정의 (Card Style) ---
\newtcolorbox{cvsection}[1]{
    breakable,
    enhanced,
    colback=white,
    colframe=navyblue,
    coltitle=white,
    fonttitle=\Large,
    title={#1},
    arc=2mm,
    boxrule=1pt,
    left=4mm, right=4mm, top=3mm, bottom=3mm,
    toptitle=2mm, bottomtitle=2mm,
    after skip=1.2em
}

% --- 문서 시작 ---
\begin{document}
\pagestyle{empty}
\begin{center}
    % \color{black}을 추가하여 이 부분만 검은색으로 고정
    \Huge \textbf{{\color{black}이력서}}
\end{center}

% [중요] 패키지 로드 부분에 \usepackage{tabularx} 와 \usepackage{array} 가 있어야 합니다.

% ====================================================================
% [헤더 영역]
% ====================================================================
\noindent
\begin{minipage}[c]{3.2cm}
    \centering
    \BLOCK{ if photo_path }
        % 사진 테두리 (선택사항)
        \setlength{\fboxsep}{0pt}
        \setlength{\fboxrule}{0.5pt}
        % [핵심] 너비 3cm, 높이 4cm로 강제 고정
        \fbox{\includegraphics[width=3cm, height=4cm]{\VAR{photo_path}}}
    \BLOCK{ else }
        % 사진 없을 때 빈 박스도 3cm x 4cm
        \framebox{\parbox[c][4cm][c]{3cm}{
            \centering\small\color{gray}사진\\(3cm$\times$4cm)
        }}
    \BLOCK{ endif }
\end{minipage}%
\hfill
% [오른쪽] 정보 테이블
\begin{minipage}[c]{\dimexpr\textwidth-3.5cm\relax}
    % [높이 맞춤 핵심]
    % 사진 높이(4cm)에 맞추기 위해 행 높이를 2.2배로 설정 (4행 기준 약 4cm)
    \renewcommand{\arraystretch}{2.2}
    
    % 세로 중앙 정렬(m) 및 가로 중앙 정렬(Y) 설정
    \renewcommand{\tabularxcolumn}[1]{m{#1}}
    \newcolumntype{Y}{>{\centering\arraybackslash}X}
    
    \begin{tabularx}{\linewidth}{|c|Y|c|Y|}
        \hline
        \textbf{성명} & \VAR{name} & \textbf{생년월일} & \VAR{birthdate} \\ \hline
        % 주소: 긴 내용 줄바꿈 허용 (너비 9.5cm)
        \textbf{주소} & \multicolumn{3}{>{\centering\arraybackslash}m{9.5cm}|}{\VAR{address}} \\ \hline
        \textbf{연락처} & \VAR{phone} & \textbf{이메일} & \small \VAR{email} \\ \hline
        \textbf{비상연락} & \multicolumn{3}{>{\centering\arraybackslash}m{9.5cm}|}{\VAR{emergency_contact}} \\ \hline
    \end{tabularx}
\end{minipage}
% ====================================================================
% [섹션 2: 경력 사항]
% ====================================================================
\begin{cvsection}{경력 사항}
    \BLOCK{ if experience and experience|length > 0 }
        \BLOCK{ for job in experience }
        \noindent 
        \begin{minipage}[t]{0.75\textwidth}
            \textbf{\large \VAR{job.company | default('OO회사')}} \\
            \textit{\small \VAR{job.role | default('일반 업무')}}
        \end{minipage}%
        \hfill
        \begin{minipage}[t]{0.20\textwidth}
            \raggedleft
            \textbf{\textit{\small \VAR{job.period | default('약 10년')}}} \\
            \textbf{\textit{\small \VAR{job.location | default('서울 지역')}}}
        \end{minipage}

        \BLOCK{ if not loop.last }
        \vspace{0.8em}
        \hrule
        \vspace{0.8em}
        \BLOCK{ endif }
        \BLOCK{ endfor }
    \BLOCK{ else }
        \noindent
        \begin{minipage}[t]{0.75\textwidth}
            \textbf{\large OO회사} \\
            \textit{\small 일반 업무}
        \end{minipage}%
        \hfill
        \begin{minipage}[t]{0.20\textwidth}
            \raggedleft
            \textbf{\textit{\small 약 10년}} \\
            \textbf{\textit{\small 서울 지역}}
        \end{minipage}
    \BLOCK{ endif }
\end{cvsection}

% ====================================================================
% [섹션 3: 학력]
% ====================================================================
\begin{cvsection}{학력}
    \BLOCK{ if education and education|length > 0 }
        \BLOCK{ for edu in education }
        \noindent
        \begin{minipage}[t]{0.75\textwidth}
            \textbf{\large \VAR{edu.institution | default('고등학교 졸업')}} \\
            \BLOCK{ if edu.degree }
                \textit{\small \VAR{edu.degree}}
            \BLOCK{ endif }
        \end{minipage}%
        \hfill
        \begin{minipage}[t]{0.20\textwidth}
            \raggedleft
            \textbf{\textit{\small \VAR{edu.period | default('1980년경')}}}
        \end{minipage}

        \BLOCK{ if not loop.last }
            \vspace{0.8em}
            \hrule
            \vspace{0.8em}
        \BLOCK{ endif }
        \BLOCK{ endfor }
    \BLOCK{ else }
        \noindent
        \begin{minipage}[t]{0.75\textwidth}
            \textbf{\large 고등학교 졸업(검정고시 포함)}
        \end{minipage}%
        \hfill
        \begin{minipage}[t]{0.20\textwidth}
            \raggedleft
            \textbf{\textit{\small 1980년경}}
        \end{minipage}
    \BLOCK{ endif }
\end{cvsection}

% ====================================================================
% [섹션 4: 자격증 및 어학]
% ====================================================================
\begin{cvsection}{자격증 및 어학}
    \BLOCK{ if certifications and certifications|length > 0 }
        % [표 형태로 변경]
        % 구조: 자격증명(6cm) | 세부내용(자동채움) | 날짜(우측정렬)
        \renewcommand{\arraystretch}{2} % 행 간격 조절
        \begin{tabularx}{\linewidth}{@{} p{6cm} X r @{}}
        \BLOCK{ for cert in certifications }
            \textbf{{\VAR{cert.title | default('관련 자격증')}}} & 
            \BLOCK{ if cert.score }
                \textbf{{\small \VAR{cert.score}}}
            \BLOCK{ elif cert.issuer }
                \textbf{{\small \VAR{cert.issuer | default('한국산업인력공단')}}}
            \BLOCK{ else }
                \textbf{{\small 한국산업인력공단}}
            \BLOCK{ endif }
            & 
            {\small\color{black}\textbf{\textit{\VAR{cert.date | default('2023년경')}}}}} \\
        \BLOCK{ endfor }
        \end{tabularx}
    \BLOCK{ else }
        \noindent
        \begin{minipage}[t]{0.75\textwidth}
            \textbf{관련 자격증} \\
            \textit{\small 한국산업인력공단}
        \end{minipage}%
        \hfill
        \begin{minipage}[t]{0.20\textwidth}
            \raggedleft
            \textbf{\textit{\small 2023년경}}
        \end{minipage}
    \BLOCK{ endif }
\end{cvsection}

% 자기소개서 메인 타이틀
\newpage
\begin{center}
    % \color{black}을 추가하여 이 부분만 검은색으로 고정
    \Huge \textbf{{\color{black}자기소개서}}
    \vspace{1em}
\end{center}

% JSON 데이터(cover_letter 리스트)를 순회하며 질문-답변 생성
\BLOCK{ if cover_letter and cover_letter|length > 0 }
    \BLOCK{ for item in cover_letter }
        % 질문(Question)을 박스 제목으로 사용
        \begin{cvsection}{\VAR{item.question | default('Q. 자기소개')}}
            % 본문 폰트 및 줄간격 설정 (가독성을 위해 조정)
            \setlength{\parindent}{0pt}  % 들여쓰기 제거
            \linespread{1.5}\selectfont  % 줄간격 1.5배 (한글 가독성)
            
            % 답변(Answer) 내용 출력
            \VAR{item.answer | default('재취업을 통해 제 경험과 노하우를 새로운 환경에서 발휘하고 싶습니다.')}
        \end{cvsection}
    \BLOCK{ endfor }
\BLOCK{ else }
    % cover_letter가 없으면 기본 자기소개서 생성
    \begin{cvsection}{Q1. 지원 동기 및 희망 직무}
        \setlength{\parindent}{0pt}
        \linespread{1.5}\selectfont
        재취업을 통해 제 경험과 노하우를 새로운 환경에서 발휘하고 싶습니다. 오랜 기간 쌓아온 전문성과 성실한 자세로 조직에 기여하겠습니다.
    \end{cvsection}
    
    \begin{cvsection}{Q2. 본인의 강점 및 일하는 스타일}
        \setlength{\parindent}{0pt}
        \linespread{1.5}\selectfont
        책임감이 강하고 꼼꼼한 성격으로 업무에 임합니다. 팀원들과의 원활한 소통을 중시하며, 주어진 업무를 성실하게 완수합니다.
    \end{cvsection}
    
    \begin{cvsection}{Q3. 기억에 남는 경험}
        \setlength{\parindent}{0pt}
        \linespread{1.5}\selectfont
        다양한 업무 경험을 통해 문제 해결 능력을 키웠습니다. 어려운 상황에서도 침착하게 대응하고, 최선의 결과를 도출하기 위해 노력합니다.
    \end{cvsection}
\BLOCK{ endif }

\end{document}